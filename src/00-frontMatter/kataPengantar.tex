%-----------------------------------------------------------------------------%
\chapter*{\kataPengantar}
%-----------------------------------------------------------------------------%
\pagestyle{first-pages}

Segala puji dan syukur kami panjatkan kepada Tuhan Yang Maha Esa atas berkat,
rahmat, dan bimbingan-Nya sehingga penulis dapat menyelesaikan penyusunan skripsi ini dengan judul ``\textit{Re-Ranker} Berbasis Fitur untuk \ir{} Dokumen Legal''. Skripsi ini disusun sebagai salah satu syarat untuk memperoleh gelar Sarjana Strata Satu (S1) pada program studi Ilmu Komputer Fakultas Ilmu Komputer Universitas Indonesia. Penulis ingin mengucapkan terima kasih kepada pihak-pihak yang sudah memberikan dukungan dan bimbingan dalam proses penyusunan skripsi ini.

Ucapan Terima Kasih:
\begin{enumerate}[topsep=0pt,itemsep=-1ex,partopsep=1ex,parsep=1ex]
	\item Orang tua, atas cinta, doa, dukungan moral, dan pengorbanannya agar saya dapat menyelesaikan pendidikan ini.
	\item Pak Alfan yang telah memberikan arahan dan bimbingan selama penulisan tugas akhir ini.
	\item Dosen yang telah memberikan ilmu dan pengetahuan selama masa studi.
	\item Instansi yang telah memberikan kesempatan dan fasilitas untuk menyelesaikan penulisan tugas akhir.
	\item Sahabat, atas semangat, dukungan, dan kebersamaannya selama masa studi.
\end{enumerate}

Penulis menyadari bahwa laporan \type~ini masih jauh dari sempurna. Oleh karena itu, apabila terdapat kesalahan atau kekurangan dalam laporan ini, Penulis memohon agar kritik dan saran bisa disampaikan langsung melalui \f{e-mail} \code{kennjthn12@gmail.com}.

% Untuk input gambar tanda tangan, silahkan sesuaikan xshift, yshift, dan width dengan gambar tanda tangan Anda
% \begin{tikzpicture}[remember picture,overlay,shift={(current page.north east)}]
% \node[anchor=north east,xshift=-3cm,yshift=-20cm]{\includegraphics[width=2.8cm]{assets/pdfs/Kenneth_TandaTangan_cropped.pdf}};
% \end{tikzpicture}

\vspace*{0.1cm}
\begin{flushright}
Depok, \tanggalSiapSidang\\[0.1cm]
\ifx\blank\npmDua
	\vspace*{1.5cm}
	\penulisSatu
\else
	Tim Penulis
\fi

\end{flushright}
