%---------------------------------------------------------------
\chapter{\babEnam}
\label{bab:6}
Setelah melakukan eksperimen dan analisis hasil, pada \bab{}~\ref{bab:6} akan dipaparkan kesimpulan penelitian dan saran penelitian berikutnya. Pada \subbab{}~\ref{subbab:kesimpulan} ditarik kesimpulan dari analisis yang telah dilakukan. Kemudian, \subbab{}~\ref{subbab:saran} membahas tentang saran untuk penelitian berikutnya.
%---------------------------------------------------------------





%---------------------------------------------------------------
\section{Kesimpulan}
\label{subbab:kesimpulan}
\subbab{} ini menyajikan kesimpulan utama dari penelitian yang telah dilakukan. Berdasarkan analisis dan hasil eksperimen, beberapa poin penting dapat disimpulkan sebagai berikut:

\vspace{2mm}
\noindent\textbf{Karakteristik Fitur}. Menurut hasil eksperimen yang didapatkan, karakteristik yang dapat membedakan jenis relevansi adalah fitur nilai relevansi \obm{}, relevansi LEMURTF\_IDF, dan kesamaan jaccard. Ketiga fitur tersebut secara konsisten mendapatkan nilai \feature{} \importance{} yang tinggi dari seluruh sistem yang diuji.

\vspace{2mm}
\noindent\textbf{Peningkatan Efektivitas}. Pemanfaatan seluruh fitur untuk \reranking{} dapat meningkatkan performa seluruh metrik sekitar $4,17\%$ secara signifikan. Sedangkan pemanfaatan himpunan bagian dari fitur yang diusulkan, yaitu ketiga fitur terpenting yang konsisten, dapat meningkatkan nilai metrik utama sekitar $3, 739\%$, namun tidak cukup signifikan saat diuji menggunakan data \testing{}.
% \begin{itemize}
%     \item Pemanfaatan seluruh fitur untuk \reranking{} dapat meningkatkan performa seluruh metrik sekitar $4,17\%$ secara signifikan;
%     \item Fitur nilai relevansi \obm{}, relevansi LEMURTF\_IDF, dan kesamaan jaccard merupakan karakteristik penting dari pasangan pertanyaan-pasal untuk melakukan penilaian relevansi;
%     \item Fitur nilai relevansi \obm{}, relevansi LEMURTF\_IDF, dan kesamaan jaccard dapat meningkatkan performa sistem sekitar $3, 739\%$ namun tidak cukup signifikan saat diuji menggunakan data \testing{};
%     \item Hasil kinerja \base{} \retriever{} tidak dapat dipastikan memiliki peningkatan efektivitas yang setara dengan ditambahkannya \reranker{} berbasis fitur walaupun menggunakan himpunan fitur yang sama;
%     \item Kinerja terbaik untuk \recall{} pada \cutoff{} 3, secara umum, didapatkan menggunakan sistem \ir{} dengan DLH13 sebagai \base{} \retriever{} dan \reranker{} \lambdamart{} menggunakan seluruh fitur yang diusulkan;
%     \item Tidak ditemukan korelasi yang kuat antara nilai \cutoff{} dan efektivitas dari sistem \cascaded{} \ir{} berdasarkan nilai korelasi \textit{pearson}.
% \end{itemize}
%---------------------------------------------------------------





%---------------------------------------------------------------
\section{Saran}
\label{subbab:saran}
Berdasarkan hasil penelitian ini, terdapat beberapa saran untuk penelitian selanjutnya:
\begin{itemize}
\item Replikasi metode yang telah diusulkan pada dokumen hukum berbahasa Indonesia;
\item Menginvestigasi jumlah fitur yang efisien untuk suatu himpunan fitur menggunakan \textit{incremental ablation learning};
\item Perlu dilakukan penelitian lebih lanjut untuk mengeksplorasi fitur-fitur tambahan yang dapat meningkatkan efektivitas klasifikasi relevansi. Penggunaan teknik pembelajaran mesin yang lebih canggih, seperti \textit{deep learning}, juga patut dipertimbangkan;
\item Penelitian mendalam terhadap kombinasi berbagai \reranker{} berbasis fitur perlu dilakukan untuk mengidentifikasi konfigurasi yang paling efektif dalam meningkatkan kinerja \retriever{};
\item Pengembangan dan pengujian model yang lebih kompleks untuk sistem \cascaded{} \ir{} dengan menambahkan tahapan \ranking{} untuk mengidentifikasi faktor-faktor yang berkontribusi terhadap efektivitas sistem;
\item Implementasi dan evaluasi metode pada skenario di luar domain legal untuk memberikan wawasan berharga mengenai kinerja sistem dalam lingkup yang beragam.
\end{itemize}
%---------------------------------------------------------------




%---------------------------------------------------------------
% Pada bab ini, Penulis akan memaparkan kesimpulan penelitian dan saran untuk penelitian berikutnya.
% \item Studi komprehensif tentang pengaruh berbagai metode \retriever{} dan \reranker{} terhadap \recall{} pada berbagai \cutoff{} akan memberikan pemahaman yang lebih baik mengenai dinamika sistem \ir{};


% %---------------------------------------------------------------
% \section{Kesimpulan}
% \label{subbab:kesimpulan}
% %---------------------------------------------------------------
% Berikut ini adalah kesimpulan terkait pekerjaan yang dilakukan dalam penelitian ini:
% \begin{enumerate}
% 	\item \bo{Poin pertama} \\
% 	Penjelasan poin pertama.
% 	\item \bo{Poin kedua} \\
% 	Penjelasan poin kedua.
% \end{enumerate}

% Tulis kalimat penutup di sini.


% %---------------------------------------------------------------
% \section{Saran}
% \label{subbab:saran}
% %---------------------------------------------------------------
% Berdasarkan hasil penelitian ini, berikut ini adalah saran untuk pengembangan penelitian berikutnya:
% \begin{enumerate}
% 	\item Saran 1.
% 	\item Saran 2.
% \end{enumerate}
