%-----------------------------------------------------------------------------%
\chapter{\babSatu}
\label{bab:1}
Praktisi hukum seperti hakim, pengacara, dan \paralegal{} sering kali harus melakukan penelitian hukum yang mendalam saat menangani kasus atau perkara. Mereka memerlukan akses cepat dan akurat ke pasal-pasal atau yurisprudensi yang relevan. Untuk mendukung kebutuhan ini, penelitian ini berfokus pada pengembangan sistem \ir{} (IR) khusus untuk dokumen peraturan perundang-undangan. Sistem ini menggunakan pendekatan \cascaded{} \retrieval{} dan teknik \reranking{} menggunakan \reranker{} berbasis fitur untuk meningkatkan efisiensi dan akurasi dalam menemukan informasi hukum yang diperlukan. Sebagai awalan dari laporan penelitian, \subbab{}~\ref{bab:1} menjelaskan motivasi, pertanyaan, dan lingkup kajian dari penelitian ini. Pertama, \subbab{}~\ref{subbab:1:Latar Belakang} menyampaikan latar belakang dilakukannya penelitian ini. Kemudian, \subbab{}~\ref{subbab:1:Definisi Permasalahan} menguraikan dua permasalahan yang menjadi fokus penelitian ini. Setelah itu, \subbab{}~\ref{subbab:1:Batasan Permasalahan} membahas beberapa batasan terkait fitur yang digunakan untuk eksperimen. Selanjutnya, \subbab{}~\ref{subbab:1:Tujuan Penelitian} mendefinisikan tujuan dari penelitian. Sehabis itu, \subbab{}~\ref{subbab:1:Langkah Penelitian} mendeskripsikan langkah-langkah secara umum yang diambil dalam penelitian. Terakhir, \subbab{}~\ref{subbab:1:Sistematika Penulisan} menjelaskan sistematika penulisan laporan ini.
%-----------------------------------------------------------------------------%





%-----------------------------------------------------------------------------%
\section{Latar Belakang}
\label{subbab:1:Latar Belakang}
% Sistem penataan peraturan perundang-undangan di Indonesia memiliki beberapa masalah terkait kuantitas yang terus bertambah dan kualitas yang kurang terjamin diindikasikan dengan inkonsistensi dan konflik antar aturan perundang-undangan yang ada~\citep{amin2020mengurai}. Menurut basis data peraturan perundang-undangan\footnote{https://peraturan.go.id/}, terdapat sekitar 58.000 peraturan dan lebih dari 3.000 peraturan diantaranya sudah tidak berlaku.  

Dokumen hukum dalam jumlah yang besar, setiap hari, akan dievaluasi maupun ditambahkan oleh pihak yang berwenang~\citep{kim2024legal}. Dengan jumlah peraturan tersebut, proses pengumpulan peraturan yang relevan akan memerlukan waktu yang signifikan. Terlebih lagi proses evaluasi pengaplikasian peraturan pada kasus tertentu, seperti menentukan hubungan atau konflik antar aturan, yang sulit untuk dilakukan secara manual~\citep{kim2024legal}. Oleh karena itu, sebagai upaya efisiensi proses pengumpulan peraturan tersebut, diusulkan suatu sistem yang dapat memanfaatkan model atau algoritma \ir{} (IR) yang merupakan solusi umum untuk permasalahan tersebut~\citep{goebel2023summary, katz2023natural, kim2024legal, nguyen2024enhancing}.

\ir{} adalah bidang studi mengenai pengorganisasian dan pengambilan informasi dari suatu kumpulan data yang besar. Konsep ini, muncul karena adanya kebutuhan memperoleh informasi relevan dalam waktu yang cepat~\citep{sanderson2012history}. Namun, tantangan yang dihadapi tidak hanya terkait dengan menemukan informasi yang relevan, tetapi juga dalam memastikan bahwa informasi tersebut akurat dan berkualitas. Oleh karena itu, untuk meningkatkan kualitas dari informasi yang dikembalikan, terdapat beberapa pendekatan yang salah satunya adalah sistem \cascaded{} \ir{}~\citep{wang2011cascade}. Sistem \cascaded{} \ir{} merupakan pendekatan proses \retrieval{} yang dilakukan dalam beberapa tahap, dengan setiap tahap menggunakan algoritma yang berbeda dan umumnya semakin kompleks untuk menyaring dan memperbaiki hasil sebelumnya~\citep{zhan2020learning}. Penelitian ini berfokus pada penerapan \reranker{} berbasis fitur dalam sistem \cascaded{} \ir{} untuk peningkatan efektivitas proses \retrieval{}. Dengan pendekatan ini, diharapkan dapat meningkatkan kualitas dari peraturan yang dikembalikan dan menganalisis karakteristik yang dapat membantu proses pencarian tersebut.
%-----------------------------------------------------------------------------%





%-----------------------------------------------------------------------------%
\section{Definisi Permasalahan}
\label{subbab:1:Definisi Permasalahan}
Seperti yang sudah disampaikan pada \subbab~\ref{subbab:1:Latar Belakang}, penelitian ini fokus kepada pengembangan model untuk \reranking{} hasil dari \sparse{} \retrieval{}, seperti \obm{}, dalam domain koleksi perundang-undangan. Pertanyaan besar yang dikaji adalah ``Bagaimana membangun model \reranking{} berbasis fitur yang efektif?'' Pertanyaan tersebut kemudian dipecah menjadi dua pertanyaan penelitian berikut:
%Berikut ini adalah rumusan permasalahan dari penelitian yang dilakukan:
\begin{itemize} [topsep=0pt, itemsep=-1ex, partopsep=1ex, parsep=1ex]
    \item Apa saja karakteristik dari pasangan kasus dan pasal relevan yang dapat membedakan jenis relevansi?
    \item Sejauh mana karakteristik tersebut dapat dimanfaatkan untuk meningkatkan efektivitas proses \retrieval{}?
\end{itemize}
%-----------------------------------------------------------------------------%





%-----------------------------------------------------------------------------%
\section{Batasan Permasalahan}
\label{subbab:1:Batasan Permasalahan}
Untuk menjawab pertanyaan yang menjadi permasalahan dalam penelitian ini, ditetapkan beberapa batasan agar penelitian menjadi lebih spesifik dan terarah. Berikut ini adalah asumsi yang digunakan sebagai batasan penelitian ini:
\begin{itemize} [topsep=0pt, itemsep=-1ex, partopsep=1ex, parsep=1ex]
    \item Model yang digunakan untuk memperoleh representasi vektor dibatasi dengan hanya menggunakan dua model \transformer{}, yaitu \lbert{} (\bert{}) dan \ttttt{} (\tfive{});
    \item Model \reranker{} berbasis fitur yang digunakan dalam penelitian ini adalah \lambdamart{}.
\end{itemize}
%-----------------------------------------------------------------------------%





%-----------------------------------------------------------------------------%
\section{Tujuan Penelitian}
\label{subbab:1:Tujuan Penelitian}
Dengan menetapkan permasalahan dan batasan penelitian, diharapkan dapat mencapai tujuan dari penelitian ini. Tujuan penelitian tersebut diuraikan sebagai berikut:
\begin{itemize} [topsep=0pt, itemsep=-1ex, partopsep=1ex, parsep=1ex]
    \item Menentukan karakteristik dari pasangan kasus dan pasal yang relevan;
    \item Meningkatkan efektivitas sistem \ir{} dokumen legal dengan implementasi \reranker{} menggunakan fitur yang telah dianalisis.
\end{itemize}
%-----------------------------------------------------------------------------%





%-----------------------------------------------------------------------------%
\section{Langkah Penelitian}
\label{subbab:1:Langkah Penelitian}
Agar dapat menjawab permasalahan dan mencapai tujuan tersebut dengan cara yang dapat direplikasi maupun divalidasi, penelitian ini akan dilaksanakan dalam tahapan yang sistematis. Tahapan tersebut akan dijelaskan secara singkat sebagai berikut:
\begin{itemize}
    \item Perumusan Masalah \\
    Langkah awal ini melibatkan identifikasi masalah yang akan diselesaikan dan penetapan tujuan penelitian secara rinci untuk mengarahkan seluruh proses penelitian.
    
    \item Tinjauan Literatur \\
    Tahap ini melibatkan kajian mendalam terhadap teori, konsep, dan fakta yang relevan dengan masalah atau tujuan penelitian untuk membantu dalam memahami konteks penelitian dan merumuskan hipotesis atau pertanyaan penelitian.
    
    \item Rancangan Penelitian \\
    Tahap ini mencakup perancangan penelitian berdasarkan hasil tinjauan literatur, meliputi strategi pengumpulan dan pemrosesan data, serta perancangan sistem, teknik analisis, dan metode evaluasi yang digunakan.
    
    \item Implementasi \\
    Pada tahap ini, dilakukan implementasi dari rancangan penelitian pada tahapan sebelumnya untuk menjawab rumusan masalah yang sudah ditentukan.
    
    \item Analisis dan Interpretasi \\
    Hasil implementasi akan dianalisis dan diinterpretasikan untuk menjawab perumusan masalah dan menyimpulkan hasil penelitian.
\end{itemize}
%-----------------------------------------------------------------------------%





%-----------------------------------------------------------------------------%
\section{Sistematika Penulisan}
\label{subbab:1:Sistematika Penulisan}
Dokumen ini disusun dengan sistematika penulisan yang teratur dan logis untuk memberikan pemahaman yang jelas dan terstruktur tentang penelitian ini. Berikut adalah sistematika penulisan yang digunakan:
% Setiap bab dalam penulisan ini dirancang untuk membawa pembaca melalui setiap tahap penelitian, mulai dari pendahuluan yang menjelaskan motivasi penelitian hingga penutup yang berisi kesimpulan dan saran. 
\begin{itemize}
    \item \bab{}~\ref{bab:1}~\babSatu \\
    Pada \bab{}~\ref{bab:1}, dijelaskan latar belakang, definisi permasalahan, batasan permasalahan, tujuan penelitian, posisi penelitian, langkah penelitian, dan sistematika penulisan laporan tugas akhir ini.
    
    \item \bab{}~\ref{bab:2}~\babDua \\
    Pada \bab{}~\ref{bab:2}, dipelajari teori yang berkaitan dengan penelitian ini untuk memperoleh konsep-konsep dasar yang diperlukan dalam mencapai tujuan dari penelitian.
    
    \item \bab{}~\ref{bab:3}~\babTiga \\
    Pada \bab{}~\ref{bab:3}, diuraikan pendekatan yang diambil saat melakukan eksperimen, meliputi alur pekerjaan, \dataset{}, usulan pengembangan model, dan evaluasi model.
    
    \item \bab{}~\ref{bab:4}~\babEmpat \\
    Pada \bab{}~\ref{bab:4}, dibahas implementasi metodologi yang telah dijelaskan pada \bab{}~\ref{bab:3} dalam bentuk pemrograman.

    \item \bab{}~\ref{bab:5}~\babLima \\
    Pada \bab{}~\ref{bab:5}, dijelaskan hasil eksperimen dan analisis yang telah dilakukan.

    \item \bab{}~\ref{bab:6}~\babEnam \\
    Pada \bab{}~\ref{bab:6}, ditarik kesimpulan dari hasil penelitian ini beserta saran untuk pengembangan atau penelitian selanjutnya.
\end{itemize}
%-----------------------------------------------------------------------------%